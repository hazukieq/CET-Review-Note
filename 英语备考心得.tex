  % !TeX document-id = {your-document-id}
%!TeX TXS-program:compile = txs:///xelatex/[-8bit -shell-escape]
\documentclass[a4paper,12pt]{ctexart}
% ==============================
% 包和设置
% ==============================
\usepackage{xeCJK}              % 支持中文字符
\usepackage{graphicx}           % 支持插入图像
\usepackage{enumitem}           % 自定义列表环境
\usepackage{tabularx}           % 支持自适应宽度的表格
\usepackage{geometry}           % 页面边距设置
\usepackage{fancyhdr}           % 自定义页眉和页脚
\usepackage{algorithm}           % 支持算法环境
%\usepackage{minted}             % 支持高亮代码
%\usepackage{listings}           % 支持代码高亮(另一种方式)
\usepackage{xcolor}             % 支持颜色
\usepackage{tcolorbox}          % 支持自定义彩色盒子
\usepackage{xeCJKfntef}         % 支持中文的脚注和尾注
\usepackage{wallpaper} % 壁纸背景设置
\usepackage{tocloft}
% 超链接设置
\usepackage[
bookmarks=true,
colorlinks=true,
colorlinks=true,
linkcolor=black,
urlcolor=red,
citecolor=green,
pdfstartview=fitH
]{hyperref}                     % 支持文档内超链接

% ==============================
% 配置字体
% ==============================
\newCJKfontfamily{\jn}{SourceHanSerifSC-Regular.otf}[Path=./fonts/]

\newCJKfontfamily{\kx}{KX.ttf}[Path=./fonts/]

\newCJKfontfamily{\wk}{LXGWWenKaiMono-Regular.ttf}[Path=./fonts/]

\setmainfont{SourceHanSerifSC-Regular.otf}[Path=./fonts/]
\setCJKmainfont{SourceHanSerifSC-Regular.otf}[Path=./fonts/]

% ==============================
% 配置标题
% ==============================
\newcommand{\makeTitle}{
	{\kx\zihao{1}\theTitle}
	
	\vspace{2em}
	\zihao{-3}\theAuthor
		
	\vspace{.5em}
	\zihao{-3}\today
}


% ==============================
% 配置CTEX文档
% ==============================
\ctexset{
	today=small,                % 日期格式
	section/name={第,章},      % 节的名称
	section/number=\chinese{section},  % 节的编号
	section/format+=\raggedright, % 节格式
	section/aftername=\hskip 0.5em, % 子节名称后空格
	subsection/aftername=\hskip 0.5em, % 子节名称后空格
	subsubsection/aftername=\hskip 0.5em, % 子子节名称后空格
}

% ==============================
% 页面边距设置
% ==============================
\geometry{
	top=2.5cm,                  % 上边距
	bottom=2.5cm,               % 下边距
	left=2cm,                   % 左边距
	right=2cm                   % 右边距
}

% 行间距设置
\setlength\headheight{52pt}   % 页眉高度
\linespread{1.3}               % 段落间距

% ==============================
% 页眉和页脚设置
% ==============================
\usepackage{fancyhdr}
\pagestyle{fancy}
\renewcommand{\sectionmark}[1]{\markright{\CTEXifname{\CTEXthesection}{}\ #1}}
\renewcommand{\subsectionmark}[1]{} % 不为 subsection 设置标记


\fancyhf{}
\fancyhead[L]{\textnormal{\kaishu \rightmark}}
%\fancyhead[R]{\textnormal{\kaishu \thepage}}
\renewcommand{\headrulewidth}{.5pt} %注意不用 \setlength
\renewcommand{\footrulewidth}{0pt}


% ==============================
% 目录配置
% ==============================
\renewcommand{\cftsecleader}{\cftdotfill{\cftdotsep}}


% ==============================
% 自定义环境
% ==============================
\newenvironment{pol}[1]{       % 自定义有序列表环境
	\begin{enumerate}[topsep=0pt,labelsep=.5em,leftmargin=\the\dimexpr 0.5em+ #1\relax,itemsep=0em,parsep=0em, partopsep=0pt,label=\arabic*.]
	}{\end{enumerate}}

\newenvironment{pul}[1]{       % 自定义无序列表环境
	\begin{itemize}[topsep=0pt,labelsep=.5em,leftmargin=\the\dimexpr 0.5em+ #1\relax,itemsep=0em,parsep=.5em]
	}{\end{itemize}}

% ==============================
% 图像文件夹设置
% ==============================
\graphicspath{{./statics/}}    % 设置图像文件存放路径

% ==============================
% 自定义命令
% ==============================
\newcommand{\pic}[1]{           % 自定义插入图像命令
	\begin{figure}[!h]
		\centering
		\includegraphics[width=4in]{#1}   % 图像宽度设置
	\end{figure}
}

\newcommand{\fpic}[1]{          % 自定义插入带框图像命令
	\begin{figure}[!h]
		\centering
		\fbox{\includegraphics[width=4.5in]{#1}} % 带框的图像
	\end{figure}
}

\definecolor{codeBg}{HTML}{f6f8fa}  % 代码背景颜色定义
%\newminted{bash}{bgcolor=codeBg,frame=leftline,framesep=0em,framerule=.1em,rulecolor=pink,fontsize=\normalsize}  % bash 代码高亮
%\newminted{latex}{bgcolor=codeBg,frame=leftline,framesep=0em,framerule=.1em,rulecolor=pink,fontsize=\normalsize} % LaTeX 代码高亮
\definecolor{yellowbg}{HTML}{FAF9DE}
\definecolor{redbg}{HTML}{FDE6E0}
\definecolor{spanbg}{HTML}{eeeeee}  % 其他背景颜色定义


\newcommand{\cbox}[1]{            % 自定义彩色盒子命令
	\tcbox[colback=spanbg, colframe=spanbg, rounded corners=all]{#1}
}

\newcommand{\emp}[1]{              % 自定义高亮命令
	\colorbox{spanbg}{#1}
}

\newcommand{\remp}[1]{
	\colorbox{redbg}{#1}
}

\newcommand{\yemp}[1]{
	\colorbox{yellowbg}{#1}
}

\newcommand{\fillblank}{
\rule[-.2pt]{3em}{.4pt}
}

\newcommand{\fillnum}[1]{
	\underline{\makebox[2.5em]{#1}}
}

\newcommand{\rlarge}[1]{
	\textcolor{red}{{\large #1}}
}

% ==============================
% 文档信息
% ==============================
\def\theTitle{英语备考心得}
\def\theAuthor{广西医科大学\quad 叶月绘梨依}

% 开始文档
\begin{document}

	\setcounter{page}{0}
	\thispagestyle{empty}
	\begin{center}
		\vspace*{0.2\textheight}
		\makeTitle
	\end{center}

	\newpage
	\tableofcontents
	\thispagestyle{empty}
	\setcounter{page}{0}
	\fancyhead[R]{}

	\newpage
	\setcounter{page}{1}
	\pagenumbering{arabic}
	\fancyhead[R]{\textnormal{\kaishu \thepage}}

	\newpage
    \section{缘由}
    \vspace{.1\textheight}
    \begin{center}
    {\zihao{-2}
     由于极度担忧某人的英语

    故连夜整理此笔记

    希望为其提供些许复习指南
    \vspace{1.5em}
    \par
    \textcolor{purple}{带着最真挚真心的祝福,谨以葉月绘梨依名义献给她}}
    \end{center}
    \newpage

    \section{前\quad 言}
	我是葉月绘梨依,来自人文社会科学院的大二学生。大二上时考过了英语四级,下半年则通过了六级考试。

	自过了四六级后,算是许久不曾接触过英语考试了。然近来有些基础较弱的同学请教于我,虽自己四六级成绩着实一般,但有些个人经验恐怕还是有些参考意义的。

	个人经验,并没有涉及很多高深理论、纷繁的做题应试技巧,恐怕更多是讲一些阅题的方法,以及如何从容写完整份试卷的心得,或者是作答时的思路。说实话,这些内容在网络上随便找亦是可以一大堆的,故我的这份也权做分享。若阅读后能有些思索或收获,我是非常高兴和开心的。

	对于读者的身份,不做任何假定,任何欲了解四六级、有志于通过四六级者,皆可以阅读这份心得。

	最后,我非常感谢那些请教的同学,是你们给了我机会去分享个人的经验心得,也是你们耐心的倾听,才让我有机会将这份备考心得逐渐补充完善。非常感谢你们的信任!

	愿每一位同学都能顺利通过大学四六级考试,并取得不错成绩!

	\begin{flushright}
		{\wk 葉月绘梨依\quad 落笔

			\today}
	\end{flushright}


	\newpage
	\section{快速阅读}
	这个题型的文章是非常长的,若要通篇读下来再找答案,恐怕不是一个明智的选择,同时也不建议这么做,因为会在这里消耗大量宝贵的时间和精力。

	\remp{这里个人推荐停留10-12分钟},时间到了后就必须进入下一部分,切不可过多留恋。

	由于文章篇幅太长了,所以我们在阅读时需要做一些标记,以便于后续快速定位,以下是做题步骤:

	\begin{pol}{3em}
		\item {\rlarge{阅读并标记}\quad 每个段落中出现的\yemp{时间、人名、数字},都需要方框圈起来。

		其次,\yemp{段落首句尾句}则需要用\underline{下划线}画出来。首尾句不要划太多,部分即可,目的是提醒\remp{每次看段落时先看首尾句,避免重读整个段落}。

		具体效果如图:
		\fpic{quickreading-extraction.png}

		接着,我们阅读题目(待匹配句子),按照\yemp{AbC}结构切割\remp{子句(标点符号前后的句子)},依次标记\yemp{关键词(名词、动词、形容词、副词)}。A是子句的开头部分,b是子句的中间部分,C是子句的尾部。\remp{关键词需要重点关注名词和动词},以及你认为便于辨认的单词,不要圈完整句话。

		例如:36. According to a management expert,work-life balance is not as simple as giving equal amounts of time to work and personal life.

		{\small 前半句,我们看到了名词\emp{management expert};

		后半句,我们看到了:名词\emp{work-life balance},形容词\emp{simple},名词\emp{amounts},名词\emp{time to work and personal life}
		}

		故这个待匹配句子的关键词是 \\\yemp{management expert, amounts, time to work and personal life}
		}

		然后我们回到文章中寻找这些关键词,可以看到\emp{B)} 就是要找的段落。

		具体效果如图:
		\fpic{quickreading-questions.jpg}

		\item {\rlarge{寻找关键词并匹配}\quad 当我们做好粗浅标记后,就可以根据待匹配句子中的关键词去文中寻找了。记住:\remp{每次只找一句话,速度尽量快},寻找过程中只看关键单词,不要尝试阅读。第一次没找到的,就下一句,保证所有的待匹配句子都寻找过。
		}

		\item {\rlarge{考虑同义词}\quad 经过前面的筛选后,总会有两三道待匹配句子无法找到相应关键词,这时候我们就需要找意思差不多的单词了。比如\yemp{reluctant}是不乐意的意思,则可以考虑\yemp{resisted(抵制), unwilling(不愿意)}等。当看到意思差不多的单词时,需要仔细阅读单词所在句子,并与待匹配句子比较。这时侯往往需要花费些阅读和思考,\textcolor{red}{强烈建议留到最后部分再做,即整份试卷都做完了且有剩余时间的时刻}。
		}

	\end{pol}

	\newpage
	\section{十五选十}
	这个题型,在四六级中应该是\textcolor{red}{比较难的了},基础较差的同学不宜在这里花费过多的时间。另一方面,这个题型占的分值也不多,更加没必要死磕了。

	\remp{个人推荐花费时间6-8分钟},若时间到了还没写完的话,建议乱填。当然乱填也是有目的地乱填,后面会讲到。

	快速完成这里的技巧是,\yemp{尝试猜测原文下划线的词性}。词性,我们粗略地分为\textcolor{red}{名词(n.)、动词(v.)、形容词(adj.)、副词(ad.)}。

	若自己基础足够的话,可将动词继续细分为原型(v.)和过去式(v-ed.)、进行时(v-ing.),其中\emp{v-ed.\&v-ing.}有时也可构成其他词性(形容词adj.或名词n.)。这类复杂情况的判断非常依赖于自身的基础,通常不建议纠结。

	以下是具体做题步骤:
	\begin{pol}{4em}
		\item {\rlarge{标注文中词性}\quad 给文中的下划线标注词性时,\yemp{我们只需关注下划线所在句子},切勿去阅读前后句子,因为这根本没必要。当对于下划线的词性不确定时,可以试着代入动词do、形容词good、副词really、名词you,与下划线前后单词连起来看是否通顺。若有多个通顺,就标记多个词性。记住,给下划线标注词性的时间不能太久,一定要果断。\remp{粗略判断标记即可,精确度不做要求}。
		\item \rlarge{标注所给单词的词性}\quad 这里非常考验大家的词汇量,\remp{不认识某些单词是很正常的},不要去纠结,通常我们先标记认识单词的词性,\textcolor{red}{陌生单词则根据其词尾来猜词性},一般而言有以下规律可以参考:
		\begin{table}[!ht]
			\centering
			\caption{常见词性}
			\begin{tabular}{ll}
				\hline
				\textbf{词性} & \textbf{词尾} \\ \hline
				名词 & -sion,-tion,-ship,-ty,-ity,-ist,-ism,-er,-or,-dom,-ance,-ence,-acy,-al,-ness,-ment \\
				形容词 & -ful,-less,-able,-ible,-y,-ive,-ish,-ious,-ous,-ic,-ical,-al \\
				副词 & -ly,-wise,-wards \\
				动词 & -ize,-ise,-ify,-fy,-en,-ate \\ \hline
			\end{tabular}
		\end{table}
	}
	\item \rlarge{词性匹配单词}\quad 接着就是\remp{尝试将单词与原文中每个下划线的词性进行匹配},建议从首个开始,每选一个就划掉一个单词,慢慢地可选单词会变得越来越少,考虑难度也会逐渐降低。比如,第一个下划线词性是形容词,那么我们就去单词区寻找形容词,逐个代入下划线处,看这个句子表达是否自然、是否完整。若完整意味着合适,就立即填入这个单词,并在单词区中划掉这个单词。每一个下划线都是这么做,直到所有下划线都填完单词为止。

	\item \rlarge{通读检查}\quad 待选择完所有单词后,我们就要将文章快速阅读,阅读时要考虑下划线所在句子的逻辑是否通畅,或者说是否可以与前后句顺利衔接,以便及时发现错误并重选单词。当然,阅读文章速度过慢的同学,可以跳过这一步,不要消耗珍贵的时间。

	具体的文中标记与单词标记,如下图:
	\fpic{fillwords-label.png}
	\\
	\framebox[\linewidth][l]{正确答案: N I F E H A L B J C}
	\end{pol}

	\newpage
	\section{阅读篇}
	仔细阅读可谓是\remp{整份试卷中重中之重},其占的分值也是最高的,同学们应该将预留充足的时间精力来应对这个部分。毫无疑问地讲,得阅读者得天下。

	每篇文章的篇幅都适中,阅读起来并不会花费过多时间,但耗费时间更多的是在读题以及纠结选择哪个答案上。而在这个部分,我们必须始终坚持一个原则:\textcolor{red}{尽量客观,多点基于原文事实的推测,少点自以为是的猜测,少点个人主观。}

	\remp{One principle: be more objective, be less subjective.}

	对于先读题再阅读文章,或者先阅读文章再读题,这个并没有很大区别,遵从个人习惯就好。

	我想关于如何提高阅读正确率的技巧等,在网上是数不胜数,毕竟我自己也看过了不少,同学们若有合适自己的方法,则可以自信地跳过这部分了。\textcolor{purple}{我只着重讲如何快速阅读原文,并尽可能地提取出我们需要的信息。}

	在此之前,我们有必要了解一些基础知识。四六级高频话题可归类于:
	\footnote{ \href{https://cet4-6.xdf.cn/202508/14832653.html}{四六级高频话题一览-新东方网}}
	\begin{pul}{3.5em}
		\item {\remp{科技与创新}
		科技类文章是四六级阅读的常客,内容涵盖人工智能、大数据、区块链、生物技术等前沿领域。这类文章通常涉及技术原理、应用场景及其对社会的影响。}

		\item {\remp{环境与可持续发展}
		气候变化、能源危机、生态保护等话题频繁出现在真题中。文章多从全球视角出发,探讨环境问题的成因、解决方案及国际合作。}

		\item {\remp{社会与文化}
		社会类文章涉及教育、就业、性别平等、移民等热点问题,文化类文章则聚焦于全球化背景下的文化冲突与融合。}

		\item {\remp{经济与商业}
		经济类文章常讨论全球经济形势、贸易战、市场趋势等,商业类文章则聚焦企业管理、创业创新、消费者行为等。}

		\item {\remp{健康与医学}
		健康类文章涉及公共卫生、心理健康、疾病预防等,医学类文章则探讨新药研发、医疗技术突破等。}
	\end{pul}

	以下是具体做题步骤:
		\begin{pol}{3.5em}
			\item {\rlarge{结合常识读原文}\quad
				首次读原文时,尤其是在读完前面一两段后,我们需要知道它讲的内容是哪个领域,即\textcolor{purple}{高频话题中的某一大类}。

				当知道某个大类时,相关的常识就可以派上用场了。如果每个段落若不能全读懂,就看着段落中某些简单句子,然后用常识去猜测这个段落可能在讲什么,知道个大概意思即可。

				这样子,通读完全文,我们若能回答以下几个问题就足够了:
				\begin{pul}{2.5em}
					\item 这篇文章是哪个领域的?大概讲些什么东西?一两句话/词语描述
					\item 这篇文章有明确表达态度或观点吗?试着圈出来
				\end{pul}
			}
			\item {\rlarge{做标记找对应}\quad
				做一道题时,\remp{圈出每个选项中的名词、动词},然后去文章中找相似或相同的单词,锁定对应句子,注意观察对应句子的主语是否一致,其表达意思与选项是否相同。对应句子没有提到的,慎选;优先选择对应句子中有相关单词或意思有关联性的选项。
				\footnote{\remp{来自卫生事业管理专业彩虹学姐的分享}}
			}
			\fpic{reading-labels.png}

			\item {\rlarge{细读题目再选}\quad 再次阅读题目,尤其是题目中的动词,避免错误选择。注意:find、learn from、want to、say about 这类是概括总结文中事实的,conclude about 此类则要求根据文中事实推出的结论。
			尽量避免曲解、误解题目表达的意思。
			}
		\end{pol}

	\newpage
	\section*{拓展}
	英语文章的文体(Writing styles)有解释类(Expository)、描述性(Descriptive)、说服性(Persuasive)、叙述性(Narrative)。详细说明,可以参考以下表格:
	\begin{table}[!h]
		\begin{tabularx}{\textwidth}{|c|X|X|X|}
			\hline
			文体 & 描述 & 具体应用 & 受众 \\
			\hline
			解释类 & 解释或阐述事物、概念或观点,提供信息和解释 & 教科书、科学写作、新闻报道、技术手册、学术论文 & 学生、学者、专业人士、普通读者 \\
			\hline
			描述性 & 通过生动的描绘和细节,描述事物、场景或人物,使读者产生视觉感受 & 描写风景、人物描写、描绘情感、旅行记、自然景观描写 & 一般读者、旅行者、文学爱好者 \\
			\hline
			说服性 & 旨在说服读者接受作者观点或行动,强调论证和逻辑 & 广告、辩论演讲、社论、政治演讲、劝说信件 & 潜在客户、政治参与者、公众 \\
			\hline
			叙述性 & 讲述故事或事件,按照时间顺序展开,强调情节和情感 & 小说、传记、回忆录、故事、游记、童话故事 & 小说爱好者、历史爱好者、儿童 \\
			\hline
		\end{tabularx}
	\end{table}
	\footnote{ \href{https://zhuanlan.zhihu.com/p/361487688}{英语文体分类-知乎}}

	\newpage
	\section{翻译}
	\remp{个人推荐花费15-20分钟}

	对于翻译,其实并没有很好的方法去应对。尤其对于文言文类的翻译时,我们会感到非常吃力。但幸好,还有一个技巧是可以起作用的,即按照意思去翻译,而不是完全按照逐词去翻译。——想必其他资料也有提及。

	对于不懂的单词,就尽量去解释。比如你不懂水坝dam,就写block water into one place(拦着水在某一处),高原plateau不懂,就写high latitude place(高纬度地方)。再者传统概念,比如孝忠、节日等,直接用拼音就好了。


	整个句子,我们尽量将其主要意思表达出来就好。中文一个长句子,英文中可以用好几个短句去表达,甚至不需要过多考虑语法、高级词汇等的使用。目的是\remp{能用英文去表达中文}。

	即,在翻译时应该关注整体,而不应该过度地去纠结某个词语或字的翻译。可以试着看一个例子:
	\\[1em]
	\hspace*{2em}中国政府十分重视环境保护。近年来,中国在减少空气、水和土壤污染上取得了显著成效。为了不断改善人们的生活环境,中国采取了一系列有效措施,包括大力发展清洁能源,改善公共交通,推广共享单车,实施垃圾分类。通过这些措施,中国的城市和农村正在绿起来、美起来。中国还积极参与国际合作,为全球环境保护做出了重要贡献。
	\\[1em]
	\noindent\framebox[\linewidth][l]{
		\begin{minipage}{0.98\linewidth}
		翻译:\\
		Chinese goverment put environment protection into a key place. Recently there is an obvious effort on reducing pollutions of air, water and soil. For increasing people's living, China has token a plenty of actual measures, includes developing clean energy, making public transportation better, promoting share bicycles, and categorizing rubbish. With above, citys and countrysides looks more nature and beautiful. Further more, China actively take part in international cooperations, and has made a huge contribution on global environment protection.
	\end{minipage}
	}

	\newpage
	\section{作文模板}
	\remp{个人推荐花费15-20分钟}

	作文是最简单的一个了,不管是啥题目,只需要把题目中的观点代入进去即可。当然,最低的起码要能看懂题目的一半,不苛求完全看懂。

	以下是一个比较简单的作文模板:
	\\[1em]
	\hspace*{2em}With the rapid development of technology, it is of great necessity for young person to improve\fillblank[观点]. The following solutions can account for this phenomenon.

	随着科技的快速发展,年轻人提高\fillblank[观点] 的必要性变得非常重要。以下解决方案可以解释这一现象
	\\[1em]
	\hspace*{2em}Firstly, there is no doubt that\fillblank[观点的支持1]. Based on a most recent survey, it is revealed that majority of successful social
	elites admit that they prefer\fillblank(doing sth). Furthermore, it is obvious that it is beneficial for common people\fillblank(to do sth)[观点的支持2]. Where there is a will, there is a way. Last but not least, no one can deny that it is high time that doing this thing. Only in this way, can we\fillblank[观点的支持3].

	首先,毫无疑问,\fillblank[观点的支持1]。根据最近的一项调查,大多数成功的社会精英承认他们更喜欢\fillblank(做某事)[观点的支持2]。此外,很明显,对普通人来说做某事是有益的。有志者事竟成。最后但同样重要的是,没有人可以否认现在是时候做这件事了。只有这样,我们才能\fillblank[观点的支持3]。
	\\[1em]
	\hspace*{2em}In a word, it is clear that we should\fillblank[观点].

	总之,显然我们应该\fillblank[观点]。

	\newpage
	\section{实战考试}

	\remp{做题顺序}
	\begin{pol}{3.5em}
		\item 进入考场前,需要反复记忆作文模板
		\item 趁着记忆还在,先写作文,把作文模板代入去写。不要纠结如何,写完就行。花费15-20分钟
		\item 写十五选十,花费6-8分钟
		\item 写快速阅读,花费10-12分钟
		\item 写翻译,花费15-20分钟
		\item 做阅读,剩余全部时间
		\item 涂答题卡
		\item 若还有盈余,则再次检查,着重快速阅读、阅读
		\item 交卷
	\end{pol}

	\vspace*{1em}
	\remp{时间统计}
	\begin{pul}{3.5em}
		\item 写作+翻译=30-40分钟
		\item 十五选十+快速阅读=16-20分钟
		\item 阅读=1小时14分-1小时
	\end{pul}

	%\newpage
	%\section{增\quad 补}
\end{document}