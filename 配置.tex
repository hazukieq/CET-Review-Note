% ==============================
% 包和设置
% ==============================
\usepackage{xeCJK}              % 支持中文字符
\usepackage{graphicx}           % 支持插入图像
\usepackage{enumitem}           % 自定义列表环境
\usepackage{tabularx}           % 支持自适应宽度的表格
\usepackage{geometry}           % 页面边距设置
\usepackage{fancyhdr}           % 自定义页眉和页脚
\usepackage{algorithm}           % 支持算法环境
%\usepackage{minted}             % 支持高亮代码
%\usepackage{listings}           % 支持代码高亮(另一种方式)
\usepackage{xcolor}             % 支持颜色
\usepackage{tcolorbox}          % 支持自定义彩色盒子
\usepackage{xeCJKfntef}         % 支持中文的脚注和尾注
\usepackage{wallpaper} % 壁纸背景设置
\usepackage{tocloft}
% 超链接设置
\usepackage[
bookmarks=true,
colorlinks=true,
colorlinks=true,
linkcolor=black,
urlcolor=red,
citecolor=green,
pdfstartview=fitH
]{hyperref}                     % 支持文档内超链接

% ==============================
% 配置字体
% ==============================
\newCJKfontfamily{\jn}{SourceHanSerifSC-Regular.otf}[Path=./fonts/]

\newCJKfontfamily{\kx}{KX.ttf}[Path=./fonts/]

\newCJKfontfamily{\wk}{LXGWWenKaiMono-Regular.ttf}[Path=./fonts/]

\setmainfont{SourceHanSerifSC-Regular.otf}[Path=./fonts/]
\setCJKmainfont{SourceHanSerifSC-Regular.otf}[Path=./fonts/]

% ==============================
% 配置标题
% ==============================
\newcommand{\makeTitle}{
	{\kx\zihao{1}\theTitle}
	
	\vspace{2em}
	\zihao{-3}\theAuthor
		
	\vspace{.5em}
	\zihao{-3}\today
}


% ==============================
% 配置CTEX文档
% ==============================
\ctexset{
	today=small,                % 日期格式
	section/name={第,章},      % 节的名称
	section/number=\chinese{section},  % 节的编号
	section/format+=\raggedright, % 节格式
	section/aftername=\hskip 0.5em, % 子节名称后空格
	subsection/aftername=\hskip 0.5em, % 子节名称后空格
	subsubsection/aftername=\hskip 0.5em, % 子子节名称后空格
}

% ==============================
% 页面边距设置
% ==============================
\geometry{
	top=2.5cm,                  % 上边距
	bottom=2.5cm,               % 下边距
	left=2cm,                   % 左边距
	right=2cm                   % 右边距
}

% 行间距设置
\setlength\headheight{52pt}   % 页眉高度
\linespread{1.3}               % 段落间距

% ==============================
% 页眉和页脚设置
% ==============================
\usepackage{fancyhdr}
\pagestyle{fancy}
\renewcommand{\sectionmark}[1]{\markright{\CTEXifname{\CTEXthesection}{}\ #1}}
\renewcommand{\subsectionmark}[1]{} % 不为 subsection 设置标记


\fancyhf{}
\fancyhead[L]{\textnormal{\kaishu \rightmark}}
%\fancyhead[R]{\textnormal{\kaishu \thepage}}
\renewcommand{\headrulewidth}{.5pt} %注意不用 \setlength
\renewcommand{\footrulewidth}{0pt}


% ==============================
% 目录配置
% ==============================
\renewcommand{\cftsecleader}{\cftdotfill{\cftdotsep}}


% ==============================
% 自定义环境
% ==============================
\newenvironment{pol}[1]{       % 自定义有序列表环境
	\begin{enumerate}[topsep=0pt,labelsep=.5em,leftmargin=\the\dimexpr 0.5em+ #1\relax,itemsep=0em,parsep=0em, partopsep=0pt,label=\arabic*.]
	}{\end{enumerate}}

\newenvironment{pul}[1]{       % 自定义无序列表环境
	\begin{itemize}[topsep=0pt,labelsep=.5em,leftmargin=\the\dimexpr 0.5em+ #1\relax,itemsep=0em,parsep=.5em]
	}{\end{itemize}}

% ==============================
% 图像文件夹设置
% ==============================
\graphicspath{{./statics/}}    % 设置图像文件存放路径

% ==============================
% 自定义命令
% ==============================
\newcommand{\pic}[1]{           % 自定义插入图像命令
	\begin{figure}[!h]
		\centering
		\includegraphics[width=4in]{#1}   % 图像宽度设置
	\end{figure}
}

\newcommand{\fpic}[1]{          % 自定义插入带框图像命令
	\begin{figure}[!h]
		\centering
		\fbox{\includegraphics[width=4.5in]{#1}} % 带框的图像
	\end{figure}
}

\definecolor{codeBg}{HTML}{f6f8fa}  % 代码背景颜色定义
%\newminted{bash}{bgcolor=codeBg,frame=leftline,framesep=0em,framerule=.1em,rulecolor=pink,fontsize=\normalsize}  % bash 代码高亮
%\newminted{latex}{bgcolor=codeBg,frame=leftline,framesep=0em,framerule=.1em,rulecolor=pink,fontsize=\normalsize} % LaTeX 代码高亮
\definecolor{yellowbg}{HTML}{FAF9DE}
\definecolor{redbg}{HTML}{FDE6E0}
\definecolor{spanbg}{HTML}{eeeeee}  % 其他背景颜色定义


\newcommand{\cbox}[1]{            % 自定义彩色盒子命令
	\tcbox[colback=spanbg, colframe=spanbg, rounded corners=all]{#1}
}

\newcommand{\emp}[1]{              % 自定义高亮命令
	\colorbox{spanbg}{#1}
}

\newcommand{\remp}[1]{
	\colorbox{redbg}{#1}
}

\newcommand{\yemp}[1]{
	\colorbox{yellowbg}{#1}
}

\newcommand{\fillblank}{
\rule[-.2pt]{3em}{.4pt}
}

\newcommand{\fillnum}[1]{
	\underline{\makebox[2.5em]{#1}}
}

\newcommand{\rlarge}[1]{
	\textcolor{red}{{\large #1}}
}